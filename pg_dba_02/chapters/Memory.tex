\chapter{The memory}
\label{ch:PGMEMORY}
The PostgreSQL memory at first sight looks simple. If compared with the complex structures implemented in 
the other DBMS to a careless reader could seem rudimentary. However, the memory and in particular the 
shared buffers implementation is complex and sophisticated. This chapter will dig down deep into the 
PostgreSQL's memory.

\section{The shared buffer}
The shared buffer is a segment allocated at cluster's startup. Its size is determined by the GUC parameter 
shared\_buffers and the size can be changed only restarting the cluster. The shared buffer is used 
to manage the data pages as seen in \ref{sec:CLUBACKEND}, which are called buffers when loaded in memory. 
Having a shared area into the RAM have also the beneficial effect of keeping the data near the CPU for 
rapid access, keeping in memory the important things and not everything. In the era of the \textit{in 
memory databases} this could seems quite obsolete but the truth is that the resources, and the money, are 
not infinite and the memory is not cheap.

\subsection{An horrible history}
\subsection{The clock sweep}
\subsection{The wal buffers}
\subsection{Memory context}

\section{The user memory}
\subsection{Work memory}
\subsection{Maintenance work memory}
\subsection{Temporary memory}

\section{Wrap up}
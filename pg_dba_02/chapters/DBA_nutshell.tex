\chapter{DBA in a nutshell}
A database administrator is a strange combination of theory and practice. A mix of strictness and loose 
rules. It's quite difficult to explain what exactly a DBA does. Working with databases requires passion, 
knowledge and a strange combination of empathy and pragmatism in order to try to understand what the 
cluster is thinking. With the commercial products the knowledge stops to the limits imposed by the vendor's 
documentations. With a free DBMS like PostgreSQL the source code's availability facilitates the knowledge 
acquisition. Also, reading the superb poetry written in C code and creates an intimate relation with the 
cold binary programs.\newline

A day in the life of a DBA does not have fixed boundaries. It can last one day or several months, for 
example if there are long procedures to run inside the well controlled maintenance windows. Every day 
is a different combination of the duties including the routine tasks, the monitoring and the proactive 
thinking. The latter duty is probably the distinctive mark between a good and DBA and a cheap professional. 
The capability of building a mental map of any possible issue projected in the near future, can make the 
difference between spending the night sleeping or working frantically before the new day begins. Of course 
when the emergency strikes there is the fourth and most important duty. The emergency handling. 

\section{Routine tasks}
Under the group of routine tasks fall the procedures which are well consolidated on the documents and in 
the DBA mind. For example, configuring the PostgreSQL's memory parameters should be something immediate. 
Any task in this category is successful if remains completely unnoticed.

\section{Monitoring}
A system without monitoring is an one way ticket to the disaster. Whether solution is used it should be 
something simple to configure and with a decent rate of false positives. Having a nagging monitor is 
exactly the same like not having at all. The important alerts will pass completely unnoticed. 


\section{Proactive thinking}

\section{Emergency handling}

\section{Failure is not an option}
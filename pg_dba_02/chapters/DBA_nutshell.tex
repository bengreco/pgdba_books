\chapter{DBA in a nutshell}
A database administrator is a strange combination of theory and practice. A mix of strictness and loose 
rules. It's quite difficult to explain what exactly a DBA does. Working with databases requires passion, 
knowledge and a strange combination of empathy and pragmatism in order to try to understand what the 
DBMS is thinking. With the commercial products the knowledge is limited by the vendor's documentations 
and is improved mostly by the personal experience. With a free DBMS like PostgreSQL the source code's 
availability facilitates the knowledge acquisition. Also, reading the superb poetry which is the source 
code, which is C language, creates an intimate relation with the cold binary programs and the 
geniuses who create them.\newline

A day in the life of a DBA does not have fixed boundaries. It can last one day or several months, for 
example if there are long procedures to run inside the well controlled maintenance windows. Every day 
is a different combination of the duties including the routine tasks, the monitoring and the proactive 
thinking. The latter is probably the distinctive mark between a good and DBA and a cheap professional. 
The capability of building a mental map for finding any possible issue, projected in the near future, can 
make the difference between spending the night sleeping or working frantically before the next day begins. 
Of course when the emergency strikes there is the fourth and most important duty. The emergency handling. 

\section{Routine tasks}
Under the group of routine tasks fall the procedures which are well consolidated on the documents and in 
the DBA mind. For example, configuring the PostgreSQL's memory parameters should be something immediate. 
Any task in this category is successful if remains completely unnoticed. Is not unlikely for the live 
systems that the routine tasks are performed in antisocial hours like Sunday night or early morning in the 
working days.

\section{Monitoring}
A system without monitoring is an one way ticket to the disaster. Whether solution is used it should be 
something simple to configure and with a decently low rate of false positives. Having a nagging monitor is 
exactly the same like not having at all. The important alerts will pass completely unnoticed. Alongside 
with the general purpose solutions like nagios there is a new interesting PostgreSQL's dedicated called 
\href{http://opm.io/}{OPM}. Configuring a passive error detector like tail\_n\_mail is a very good idea to 
trap any possible server's misbehaviour. 

\section{Proactive thinking}
Reacting to the issues is fine. Trying to prevent them is much better. This kind of tasks is strictly 
related with the monitoring. For example let's consider a master and a slave configured in continuous 
recovery. Because the slave is made copying the data files we should expect both having the same size. 
A (not so) small size difference between the master and the standby server can tell us the filesystem have 
some problems or is configured not correctly on one box. In this example, assuming the filesystem is xfs a 
bloating box is caused by the 
\href{
http://serverfault.com/questions/406069/why-are-my-xfs-filesystems-suddenly-consuming-more-space-and-full-of
-sparse-file}{dynamic EOF allocation} introduced in the newer releases.

\section{Emergency handling}
Shit happens, deal with it. Sooner or later a disaster will strike requiring all the experience and 
competence of the database experts for putting back the system in production state. It doesn't matter if the 
issue is caused by an user screwing up the database or a power outage or a hardware failure or a fleet of 
Vogon spaceships ready to demolish the earth. The rule number zero when in emergency is \textit{never 
guess}. Guessing is a one way ticket to the disaster, in particular if what is the action 
can destroy the database.\newline

As example let's consider one of the most common causes of outage in the linux world, the distribution's 
upgrade. Usually when a linux distribution is upgraded, is very likely PostgreSQL will upgrade the 
binaries to a different major version. Now the data area is not compatible across the major releases. The 
immediate effect when a system is upgraded is PostgreSQL will refuse to start because the version's 
mismatch. The fix is quite simple. A dump and reload of the databases in the data area, or alternatively 
the upgrade in place with pg\_upgrade, will get the cluster in production state. Few years ago a 
desperate user posted a message asking for help on a message board. One ``expert'' guessed 
pg\_resetxlog would be the solution. Before reading further spend two minutes to  read the 
pg\_resetxlog 
\href{http://www.postgresql.org/docs/9.3/static/app-pgresetxlog.html}{
http://www.postgresql.org/docs/9.3/static/app-pgresetxlog.html} documentation. In the emergency, the 
``expert'' guess would transform a simple version's mismatch in a corrupted data area.

\section{Failure is not an option}
The failure is not an option. Despite this statement is quite pretentious is also the rule number 
zero of any decent DBA. The task failure, should this be a simple alter table or an emergency 
restore, is not acceptable. The database is the core of any application and therefore is the most important 
element of the infrastructure.\newline

In order to achieve this impossible level, any task should be considered single shot. Everything must run 
perfectly at the first run like if without rollback plan. However the rollback procedure must be prepared 
and tested alongside with the main task on the test environment in order to get a smooth transition if the 
rollback should happen. It's also very important to remember the checklist. This allow the DBA to catch the 
exact point of any possible failure ensuring a rapid fix when this happens.\newline

Having a plan B gives a multiple approach for the task if something goes wrong. For example, when dealing 
with normal size databases\footnote{I consider normal sized any database under the 500 GB} is possible to 
implement many strategies for the disaster recovery having a similar time needed for the recovery. When the 
amount of data becomes important this is no longer true. For example a logical restore takes more time than 
a point in time recovery or a standby failover. In this case the plan A is the physical restore.If this does 
not work then the logical recovery should be used instead.


\chapter{Restore}
\label{cha:RESTORE}
There's little advantage in saving the data if the restore is not possible. In this chapter we'll take a 
look to the fastest and possibly the safest way to restore the saved dump.\newline

The program used for the restore is determined by the dump format. We'll first take a look to the restore 
using a plain format then the custom and  the directory formats. Finally we'll the way to improve 
the restore performances with a temporary sacrifice of the cluster's reliability.

\section{The plain format}
\label{sec:PLAINFORMAT}
As seen in \ref{cha:BACKUP} the pg\_dump's output is plain SQL. The generated script gives no choice but 
loading it into psql. The SQL statements are parsed and executed in sequence.\newline

This format have few advantages. For example it's possible to edit the statements using a common 
text editor. This of course if the dump is reasonably small. Even loading a file with vim when its size is 
measured in gigabytes becomes a stressful experience though.\newline

The data contents are saved using the COPY command. At restore time this choice have the best performance.

It's possible to save the data contents using the inserts. The restore is indeed very slow because each 
statement has to be parsed, planned and executed.\newline

If the backup saves the schema and the data in two separate files this requires extra care at dump time if 
there are triggers and foreign keys in the database schema. 

The data only backup should include the switch --disable-triggers which writes emit the DISABLE 
TRIGGER statements before the data load and the ENABLE TRIGGER after the data is restored.\newline

The following example shows a dump/reload session using the separate schema and data dump files.

Let's create a new database with a simple data structure. Two tables storing a city and the address 
and a foreign key between them enforcing the referential integrity.

\begin{lstlisting}[style=pgsql]
postgres=# CREATE DATABASE db_addr;
CREATE DATABASE
postgres=# \c db_addr 
You are now connected to database "db_addr" as user "postgres".
db_addr=# CREATE TABLE t_address
        (
                i_id_addr serial,
                i_id_city integer NOT NULL,
                t_addr text,
                CONSTRAINT pk_id_address PRIMARY KEY (i_id_addr)
        )
;
CREATE TABLE
db_addr=# CREATE TABLE t_city
        (
                i_id_city       serial,
                v_city          character varying (255),
                v_postcode      character varying (20),
                CONSTRAINT pk_i_id_city PRIMARY KEY (i_id_city)
        )
;
CREATE TABLE
db_addr=# ALTER TABLE t_address ADD 
	    CONSTRAINT fk_t_city_i_id_city FOREIGN KEY (i_id_city)  
	    REFERENCES t_city(i_id_city) 
	      ON DELETE CASCADE 
	      ON UPDATE RESTRICT;
ALTER TABLE

\end{lstlisting}

Now let's put some data into the tables.

\begin{lstlisting}[style=pgsql]
INSERT INTO t_city
	    ( 
	      v_city,	
	      v_postcode
	     )
      VALUES
	    (
	      'Leicester - Stoneygate',
	      'LE2 2BH'
	    )
      RETURNING i_id_city
;

 i_id_city 
-----------
         3
(1 row)

db_addr=# INSERT INTO t_address
	    ( 
	      i_id_city,	
	      t_addr
	     )
      VALUES
	    (
	      3,
	      '4, malvern road '
	    )
      RETURNING i_id_addr
;

 i_id_addr 
-----------
         1
(1 row)


\end{lstlisting}

We'll now execute dump the schema and the data in two separate plain files. Please note we are not using 
the --disable-triggers switch.

\begin{verbatim}
postgres@tardis:~/dmp$ pg_dump --schema-only db_addr > db_addr.schema.sql
postgres@tardis:~/dmp$ pg_dump --data-only db_addr > db_addr.data.sql

\end{verbatim}

Looking to the schema dump it's quite obvious what it does. All the DDL are saved in the correct 
order to restore the same database structure .\newline
The data is then saved by pg\_dump in the correct order for having the referential integrity 
guaranteed. In our very simple example the table t\_city is dumped before the table t\_address. 
This way the data will not violate the foreign key. In a complex scenario where multiple foreign keys are 
referring the same table, the referential order is not guaranteed. Let's run the same 
dump with the option --disable-trigger.

\begin{verbatim}
postgres@tardis:~/dmp$ pg_dump --disable-triggers --data-only db_addr > db_addr.data.sql

\end{verbatim}

The copy statements in this case are enclosed by two extra statements which disable and then re enable the 
triggers.

\begin{lstlisting}[style=pgsql]

ALTER TABLE t_address DISABLE TRIGGER ALL;

COPY t_address (i_id_addr, i_id_city, t_addr) FROM stdin;
1	3	4, malvern road 
\.


ALTER TABLE t_address ENABLE TRIGGER ALL;

\end{lstlisting}

The foreign keys and all the user defined trigger will not fire during the data restore, ensuring 
the data will be safely stored and improving the speed.\newline

Let's then create a new database where we'll restore the dump starting from the saved schema.

\begin{lstlisting}[style=pgsql]
postgres=# CREATE DATABASE db_addr_restore;
CREATE DATABASE
postgres=# \c db_addr_restore 
You are now connected to database "db_addr_restore" as user "postgres".
db_addr_restore=# \i db_addr.schema.sql 
SET
...
SET
CREATE EXTENSION
COMMENT
SET
SET
CREATE TABLE
ALTER TABLE
CREATE SEQUENCE
ALTER TABLE
ALTER SEQUENCE
CREATE TABLE
ALTER TABLE
CREATE SEQUENCE
ALTER TABLE
ALTER SEQUENCE
ALTER TABLE
...
ALTER TABLE
REVOKE
REVOKE
GRANT
GRANT
db_addr_restore=# \i db_addr.data.sql 
SET
...
SET
ALTER TABLE
...
ALTER TABLE
 setval 
--------
      1
(1 row)

 setval 
--------
      3
(1 row)

db_addr_restore=# \d
                   List of relations
 Schema |          Name           |   Type   |  Owner   
--------+-------------------------+----------+----------
 public | t_address               | table    | postgres
 public | t_address_i_id_addr_seq | sequence | postgres
 public | t_city                  | table    | postgres
 public | t_city_i_id_city_seq    | sequence | postgres
(4 rows)

\end{lstlisting}



\section{The binary formats}
\label{sec:PGDUMPBINFMT}
The three binary formats supported by pg\_dump are the custom, the directory and the tar format. 
The first two support the selective access when restoring and the parallel execution. Those features 
make them the best choice for a flexible and reliable backup. Before the the 9.3 the only format supporting 
the parallel restore was the custom format. The latest version extended the functionality to the directory 
format which, combined with the parallel dump improves massively the recovery performances on big amount of 
data. The tar format which its limitations is suitable for saving only small amount of data. \newline

The custom format is a binary archive. It have a table of contents which can address the the data saved 
inside the archive. The directory format is composed by toc.dat file where the schema is stored alongside 
with the references to the zip files where the table's contents are saved. For each table there is a gz 
mapped inside the toc. Each file contains  command, COPY  or inserts, for reloading the data in the 
specific table.\newline


The restore from the binary happens via the pg\_restore program which have almost the same switches 
as pg\_dump's as seen in \ref{sec:PGDUMP}. This is the pg\_restore's help output.
\newline 

\begin{verbatim}
pg_restore restores a PostgreSQL database from an archive created by pg_dump.

Usage:
  pg_restore [OPTION]... [FILE]

General options:
  -d, --dbname=NAME        connect to database name
  -f, --file=FILENAME      output file name
  -F, --format=c|d|t       backup file format (should be automatic)
  -l, --list               print summarized TOC of the archive
  -v, --verbose            verbose mode
  -V, --version            output version information, then exit
  -?, --help               show this help, then exit

Options controlling the restore:
  -a, --data-only              restore only the data, no schema
  -c, --clean                  clean (drop) database objects before recreating
  -C, --create                 create the target database
  -e, --exit-on-error          exit on error, default is to continue
  -I, --index=NAME             restore named index
  -j, --jobs=NUM               use this many parallel jobs to restore
  -L, --use-list=FILENAME      use table of contents from this file for
                               selecting/ordering output
  -n, --schema=NAME            restore only objects in this schema
  -O, --no-owner               skip restoration of object ownership
  -P, --function=NAME(args)    restore named function
  -s, --schema-only            restore only the schema, no data
  -S, --superuser=NAME         superuser user name to use for disabling triggers
  -t, --table=NAME             restore named table(s)
  -T, --trigger=NAME           restore named trigger
  -x, --no-privileges          skip restoration of access privileges (grant/revoke)
  -1, --single-transaction     restore as a single transaction
  --disable-triggers           disable triggers during data-only restore
  --no-data-for-failed-tables  do not restore data of tables that could not be
                               created
  --no-security-labels         do not restore security labels
  --no-tablespaces             do not restore tablespace assignments
  --section=SECTION            restore named section (pre-data, data, or post-data)
  --use-set-session-authorization
                               use SET SESSION AUTHORIZATION commands instead of
                               ALTER OWNER commands to set ownership

Connection options:
  -h, --host=HOSTNAME      database server host or socket directory
  -p, --port=PORT          database server port number
  -U, --username=NAME      connect as specified database user
  -w, --no-password        never prompt for password
  -W, --password           force password prompt (should happen automatically)
  --role=ROLENAME          do SET ROLE before restore

If no input file name is supplied, then standard input is used.

Report bugs to <pgsql-bugs@postgresql.org>.
 
\end{verbatim}


pg\_restore requires a file to process and an optional database connection. If the latter is omitted 
the output is sent to the standard output. However, the switch -f specifies a file where to send the 
output instead of the standard output. This is very useful if we want just check the original dump file 
can be read, for example, redirecting the output to /dev/null.\newline

The speed of a restoring from custom or directory, using a database connection, can be massively improved 
on  a multi core system with the -j switch which specifies the number of parallel jobs to run when 
restoring the data and the post data section. \newline

As said before PostgreSQL does not supports the multithreading. Therefore each parallel job will use just 
only one cpu over a list of obects to restore determined when pg\_resotore is started.\newline

The switch --section works the same way as for pg\_dump controlling the section of the archived data to 
restore. The custom and directory format have these sections. 
\begin{itemize}
 \item \textbf{pre-data} This section contains the schema definitions without the keys, indices  and 
triggers.
\item  \textbf{data} This section contains the tables's data contents.
\item  \textbf{post-data} This section contains the objects enforcing the data integrity alongside with the 
triggers and the indices.
\end{itemize}

The switch -C is used to create the target database before starting the restore. The connections need 
also a generic database to connect in order to create the database listed in the archive. \newline

The following example shows the restore of the database created in \ref{sec:PLAINFORMAT} using the 
custom format, using the schema and the data restore.\newline

First we'll do a pg\_dump in custom format.

\begin{verbatim}
postgres@tardis:~/dump$ pg_dump -Fc -f db_addr.dmp  db_addr
pg_dump: reading schemas
pg_dump: reading user-defined tables
pg_dump: reading extensions
pg_dump: reading user-defined functions
pg_dump: reading user-defined types
pg_dump: reading procedural languages
pg_dump: reading user-defined aggregate functions
pg_dump: reading user-defined operators
pg_dump: reading user-defined operator classes
pg_dump: reading user-defined operator families
pg_dump: reading user-defined text search parsers
pg_dump: reading user-defined text search templates
pg_dump: reading user-defined text search dictionaries
pg_dump: reading user-defined text search configurations
pg_dump: reading user-defined foreign-data wrappers
pg_dump: reading user-defined foreign servers
pg_dump: reading default privileges
pg_dump: reading user-defined collations
pg_dump: reading user-defined conversions
pg_dump: reading type casts
pg_dump: reading table inheritance information
pg_dump: reading event triggers
pg_dump: finding extension members
pg_dump: finding inheritance relationships
pg_dump: reading column info for interesting tables
pg_dump: finding the columns and types of table "t_address"
pg_dump: finding default expressions of table "t_address"
pg_dump: finding the columns and types of table "t_city"
pg_dump: finding default expressions of table "t_city"
pg_dump: flagging inherited columns in subtables
pg_dump: reading indexes
pg_dump: reading indexes for table "t_address"
pg_dump: reading indexes for table "t_city"
pg_dump: reading constraints
pg_dump: reading foreign key constraints for table "t_address"
pg_dump: reading foreign key constraints for table "t_city"
pg_dump: reading triggers
pg_dump: reading triggers for table "t_address"
pg_dump: reading triggers for table "t_city"
pg_dump: reading rewrite rules
pg_dump: reading large objects
pg_dump: reading dependency data
pg_dump: saving encoding = UTF8
pg_dump: saving standard_conforming_strings = on
pg_dump: saving database definition
pg_dump: dumping contents of table t_address
pg_dump: dumping contents of table t_city

\end{verbatim}

We'll then create a new database for restoring the archive.

\begin{lstlisting}[style=pgsql]
postgres=# CREATE DATABASE db_addr_restore_bin;
CREATE DATABASE

\end{lstlisting}

The schema restore is done with using the following command.

\begin{verbatim}
postgres@tardis:~/dump$ pg_restore -v -s -d db_addr_restore_bin db_addr.dmp 
pg_restore: connecting to database for restore
pg_restore: creating SCHEMA public
pg_restore: creating COMMENT SCHEMA public
pg_restore: creating EXTENSION plpgsql
pg_restore: creating COMMENT EXTENSION plpgsql
pg_restore: creating TABLE t_address
pg_restore: creating SEQUENCE t_address_i_id_addr_seq
pg_restore: creating SEQUENCE OWNED BY t_address_i_id_addr_seq
pg_restore: creating TABLE t_city
pg_restore: creating SEQUENCE t_city_i_id_city_seq
pg_restore: creating SEQUENCE OWNED BY t_city_i_id_city_seq
pg_restore: creating DEFAULT i_id_addr
pg_restore: creating DEFAULT i_id_city
pg_restore: creating CONSTRAINT pk_i_id_city
pg_restore: creating CONSTRAINT pk_id_address
pg_restore: creating FK CONSTRAINT fk_t_city_i_id_city
pg_restore: setting owner and privileges for DATABASE db_addr
pg_restore: setting owner and privileges for SCHEMA public
pg_restore: setting owner and privileges for COMMENT SCHEMA public
pg_restore: setting owner and privileges for ACL public
pg_restore: setting owner and privileges for EXTENSION plpgsql
pg_restore: setting owner and privileges for COMMENT EXTENSION plpgsql
pg_restore: setting owner and privileges for TABLE t_address
pg_restore: setting owner and privileges for SEQUENCE t_address_i_id_addr_seq
pg_restore: setting owner and privileges for SEQUENCE OWNED BY t_address_i_id_addr_seq
pg_restore: setting owner and privileges for TABLE t_city
pg_restore: setting owner and privileges for SEQUENCE t_city_i_id_city_seq
pg_restore: setting owner and privileges for SEQUENCE OWNED BY t_city_i_id_city_seq
pg_restore: setting owner and privileges for DEFAULT i_id_addr
pg_restore: setting owner and privileges for DEFAULT i_id_city
pg_restore: setting owner and privileges for CONSTRAINT pk_i_id_city
pg_restore: setting owner and privileges for CONSTRAINT pk_id_address
pg_restore: setting owner and privileges for FK CONSTRAINT fk_t_city_i_id_city

\end{verbatim}

The dump file is specified as last parameter. The -d switch tells pg\_restore in which database restore 
the archive. Because we are connecting locally and with the postgres os user, there is no need to specify 
the authentication parameters.\newline 

Then we are ready to load the data. We'll disable again the triggers in order to avoid potential data 
load failures as seen in \ref{sec:PLAINFORMAT}.

\begin{verbatim}
postgres@tardis:~/dump$ pg_restore --disable-triggers -v -a -d db_addr_restore_bin db_addr.dmp 
pg_restore: connecting to database for restore
pg_restore: disabling triggers for t_address
pg_restore: processing data for table "t_address"
pg_restore: enabling triggers for t_address
pg_restore: executing SEQUENCE SET t_address_i_id_addr_seq
pg_restore: disabling triggers for t_city
pg_restore: processing data for table "t_city"
pg_restore: enabling triggers for t_city
pg_restore: executing SEQUENCE SET t_city_i_id_city_seq
pg_restore: setting owner and privileges for TABLE DATA t_address
pg_restore: setting owner and privileges for SEQUENCE SET t_address_i_id_addr_seq
pg_restore: setting owner and privileges for TABLE DATA t_city
pg_restore: setting owner and privileges for SEQUENCE SET t_city_i_id_city_seq
 
\end{verbatim}

The problem with this approach is the presence of the indices when loading the data which is a massive 
bottleneck. Using the --section instead of the schema and data reload improves the restore 
performance.\newline

The pre-data section will restore just the bare relations.
\begin{verbatim}
postgres@tardis:~/dump$ pg_restore --section=pre-data -v  -d db_addr_restore_bin db_addr.dmp 
pg_restore: connecting to database for restore
pg_restore: creating SCHEMA public
pg_restore: creating COMMENT SCHEMA public
pg_restore: creating EXTENSION plpgsql
pg_restore: creating COMMENT EXTENSION plpgsql
pg_restore: creating TABLE t_address
pg_restore: creating SEQUENCE t_address_i_id_addr_seq
pg_restore: creating SEQUENCE OWNED BY t_address_i_id_addr_seq
pg_restore: creating TABLE t_city
pg_restore: creating SEQUENCE t_city_i_id_city_seq
pg_restore: creating SEQUENCE OWNED BY t_city_i_id_city_seq
pg_restore: creating DEFAULT i_id_addr
pg_restore: creating DEFAULT i_id_city
pg_restore: setting owner and privileges for DATABASE db_addr
pg_restore: setting owner and privileges for SCHEMA public
pg_restore: setting owner and privileges for COMMENT SCHEMA public
pg_restore: setting owner and privileges for ACL public
pg_restore: setting owner and privileges for EXTENSION plpgsql
pg_restore: setting owner and privileges for COMMENT EXTENSION plpgsql
pg_restore: setting owner and privileges for TABLE t_address
pg_restore: setting owner and privileges for SEQUENCE t_address_i_id_addr_seq
pg_restore: setting owner and privileges for SEQUENCE OWNED BY t_address_i_id_addr_seq
pg_restore: setting owner and privileges for TABLE t_city
pg_restore: setting owner and privileges for SEQUENCE t_city_i_id_city_seq
pg_restore: setting owner and privileges for SEQUENCE OWNED BY t_city_i_id_city_seq
pg_restore: setting owner and privileges for DEFAULT i_id_addr
pg_restore: setting owner and privileges for DEFAULT i_id_city
 
\end{verbatim}

The data section will then load the data contents as fast as possible.

\begin{verbatim}
postgres@tardis:~/dump$ pg_restore --section=data -v  -d db_addr_restore_bin db_addr.dmp 
pg_restore: connecting to database for restore
pg_restore: implied data-only restore
pg_restore: processing data for table "t_address"
pg_restore: executing SEQUENCE SET t_address_i_id_addr_seq
pg_restore: processing data for table "t_city"
pg_restore: executing SEQUENCE SET t_city_i_id_city_seq
pg_restore: setting owner and privileges for TABLE DATA t_address
pg_restore: setting owner and privileges for SEQUENCE SET t_address_i_id_addr_seq
pg_restore: setting owner and privileges for TABLE DATA t_city
pg_restore: setting owner and privileges for SEQUENCE SET t_city_i_id_city_seq

\end{verbatim}

Finally, the post-data section will create the constraint and the indices over the existig data.

\begin{verbatim}
postgres@tardis:~/dump$ pg_restore --section=post-data -v  -d db_addr_restore_bin db_addr.dmp 
pg_restore: connecting to database for restore
pg_restore: creating CONSTRAINT pk_i_id_city
pg_restore: creating CONSTRAINT pk_id_address
pg_restore: creating FK CONSTRAINT fk_t_city_i_id_city
pg_restore: setting owner and privileges for CONSTRAINT pk_i_id_city
pg_restore: setting owner and privileges for CONSTRAINT pk_id_address
pg_restore: setting owner and privileges for FK CONSTRAINT fk_t_city_i_id_city

\end{verbatim}


\section{Restore performances}

When the disaster happens the main goal is to get the database up and running as fast as possible. 
Usually the data section's restore, if saved with the COPY is usually fast.\newline 

The bottleneck is the post-data section which requires CPU intensive operations with random disk access 
operations. In large databases this section can require more time than the entire data section even 
if the objects built by the post-data section are smaller. The parallel can improve the speed but 
sometimes is not enough. \newline

The postgresql.conf file can be tweaked in order to improve dramatically restore's speed up to the 40\% 
compared to production's configuration. This is possible because the restore configuration disables some 
settings used by PostgreSQL to guarantee the durability. The emergency configuration must be swapped with 
the production settings as soon as the restore is complete to avoid a further data loss. What follows 
assumes the production's database is lost and restore is reading from a custom format's backup. 

\subsection{shared\_buffers}
When Bulk load operations like reloading a big amount of data into the database generates an high eviction 
rate from the shared buffer without caching. Reducing the size of the shared buffer will not affect the 
load. Therefore more memory will be available for the backends when processing the post-data section. 
There's no fixed rule for the sizing. A gross approximation could at least 10 MB for each parallel job 
with a minimum shared buffer's size of 512 MB.   

\subsection{wal\_level}
The parameter wal\_level sets the amount of redo records to store in the WAL segments. 
By default is the value is to minimal which is used for the standalone clusters. If the cluster feeds a 
standby server or there is a the point in time recovery setup, the wal\_level must be at least archive 
or hot\_standby. If there is a PITR or a standby server available for the recover stop reading this book 
and act immediately. Restoring from a physical backup is several time faster rather a logical restore. 
Otherwise, the standby or PITR snapshot is lost as well, before starting the reload the wal\_level must be 
set to minimal to reduce the WAL generation rate.

\subsection{fsync}
Turning off fsync can improve the restore's speed. Unless there is a backup battery on the disk 
cache, turning off this parameter in production is not safe and can lead to data loss in case of power
failure. 

\subsection{checkpoint\_segments, checkpoint\_timeout}
The checkpoint is a periodic event in the database activity. When occurs all the dirty pages in the 
shared buffer are synced to disk. This can interfere with the restore's bulk operations. Increasing the
checkpoint segments and the checkpoint timeout to the maximum allowed will limit the unnecessary IO caused 
by the frequent checkpoints.


\subsection{autovacuum}
Turning off the autovacuum will avoid to have the tables meanwhile are restored reducing the unnecessary 
IO. 


\subsection{max\_connections}
Reducing the max connections to the number of restore jobs plus an extra headroom of five connections will 
limit the memory consumption caused by the unused memory slots. 

\subsection{port and listen\_addresses}
When the database is restoring nobody except pg\_restore and the DBA should connect it. Changing the port 
to a different value and disabling the listen addresses except the localhost is a quick and easy 
solution to avoid the users messing up with the restore process.

\subsection{maintenance\_work\_memory}
This parameter affects the post-data section's speed. With low values the backends will sort the data on 
disk slowing down the restore. Higher values will build the new indices in memory improving the speed. 
However the value's size should be set keeping in mind how much ram is available on the system. This value 
should be reduced by a 20\% if the total ram lesser than to 10 GB. For bigger amount of memory the 
reduction should be the 10\%. This will leave out from the estimate the memory consumed by the operating 
system and the other processes. From the remaining memory ram scould be removed the the shared\_buffer's 
memory. Finally the remaining value must be divided by the max connections.\newline

Let's consider a a system with 26GB ram. If we set the shared\_buffer to 2 GB and 10 max connections, the 
maintenance\_work\_mem will be 2.14 GB.
\newpage
\begin{verbatim}
26 - 10% =  23.4
23.4 - 2 = 21.4
21.4 / 10 = 2.14
\end{verbatim}


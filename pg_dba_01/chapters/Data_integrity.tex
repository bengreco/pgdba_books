\chapter{Data integrity}
\label{cha:DATAINT}\index{data integrity}
There is only one thing worse than losing the database. When the data is rubbish. In this chapter we'll
have a brief look to the constraints available in PostgreSQL and how they can be used to preserve the
data integrity..\newline

A constraint\index{constraint}, like the name suggest enforces one or more restrictions over the
table's data. When the restriction enforces the data on the relation where the constraint is defined,
then the constraint is local. When the constraints validates the local using the data in a different
relation then the constraint is foreign.\newline

The constraints can be defined like table or column constraint. The table constraints are defined at
table level, just after the field's list. A column constraint is defined in the field's definition after
the data type.\newline

When a constraint is created the enforcement applies immediately. At creation time the table's data is
validated against the constraint. If any validation error, then the creation aborts. However, the foreign
keys and the check constraints can be created without the initial validation using the clause NOT
VALID\index{constraint, NOT VALID}. This clause tells PostgreSQL to not validate the constraint's
enforcement on the existing data, improving the creation's speed.

\section{Primary keys} 
A primary key enforces the uniqueness of the participating fields. The uniqueness is enforced at
the strictest level because even the NULL values are not permitted. The primary key creates an implicit
unique index on the key's fields. Because the index creation requires the read lock this can cause
downtime. In \ref{sec:REINDEX} is explained a method which in some cases helps to minimise the
disruption. A table can have only one primary key.\newline

A primary key can be defined with the table or column constraint's syntax.

\begin{lstlisting}[style=pgsql]

--PRIMARY KEY AS TABLE CONSTRAINT
CREATE TABLE t_table_cons
        (
                i_id            serial,
                v_data          character varying (255),
                CONSTRAINT pk_t_table_cons PRIMARY KEY (i_id)
        )
;


--PRIMARY KEY AS COLUMN CONSTRAINT
CREATE TABLE t_column_cons
        (
                i_id            serial PRIMARY KEY,
                v_data          character varying (255)
        )
;
 

\end{lstlisting}

With the table constraint syntax it's possible to specify the constraint name.\newline

The previous example shows the most common primary key implementation. The constraint is defined over a
serial field. The serial\index{serial} type is a shortcut for \textbf{integer NOT NULL} with the default
value set by an auto generated sequence. The sequence's upper limit is 9,223,372,036,854,775,807.
However the integer's upper limit is just 2,147,483,647. On tables with a high generation for the key's
new values the bigserial\index{bigserial} should be used instead of serial. Changing the field's
type is still possible but unfortunately this requires a complete table's rewrite.\newline 


\begin{lstlisting}[style=pgsql]
postgres=# SET client_min_messages='debug5';
postgres=# ALTER TABLE t_table_cons ALTER COLUMN i_id SET DATA TYPE  bigint; 
DEBUG:  StartTransactionCommand
DEBUG:  StartTransaction
DEBUG:  name: unnamed; blockState:       DEFAULT; state: INPROGR, xid/subid/cid: 0/1/0, nestlvl: 1,
children: 
DEBUG:  ProcessUtility
DEBUG:  drop auto-cascades to index pk_t_table_cons
DEBUG:  rewriting table "t_table_cons"
DEBUG:  building index "pk_t_table_cons" on table "t_table_cons"
DEBUG:  drop auto-cascades to type pg_temp_51718
DEBUG:  drop auto-cascades to type pg_temp_51718[]
DEBUG:  CommitTransactionCommand
DEBUG:  CommitTransaction
DEBUG:  name: unnamed; blockState:       STARTED; state: INPROGR, xid/subid/cid: 9642/1/14, nestlvl: 1,
children: 
ALTER TABLE


\end{lstlisting}

The primary keys can be configured as natural keys, with the field's values meaningful in the real
world. For example a table storing the cities will have the field v\_city as primary key instead of
the surrogate key i\_city\_id.                                                                          
        

\begin{lstlisting}[style=pgsql]
--PRIMARY NATURAL KEY 
CREATE TABLE t_cities
        (
                v_city          character varying (255),
                CONSTRAINT pk_t_cities PRIMARY KEY (v_city)
        )
;
\end{lstlisting}

This results in a more compact table with the key values already indexed.


\section{Unique keys}
The unique keys are very similar to the primary keys. They enforce the uniqueness using an implicit 
index but the NULL values are permitted. Actually a primary key is the combination of a unique key
and the NOT NULL constraint. The unique keys are useful when the uniqueness should be enforced on
fields not participating to the primary key. 
\newpage

\section{Foreign keys}
\label{sec:FKEYS}
A foreign key is a constraint which ensures the data is compatible with the values stored in a foreign
table's field. The typical example is when two tables need to enforce a relationship. For example let's
consider a table storing the addresses.

\begin{lstlisting}[style=pgsql]
CREATE TABLE t_addresses
        (
                i_id_address    serial,
                v_address       character varying(255),
                v_city          character varying(255),
                CONSTRAINT pk_t_addresses PRIMARY KEY (i_id_address)
        )
;
\end{lstlisting}

Being the city a value which can be the same for many addresses is more efficient to store the city name
into a separate table and set a relation to the address table.

\begin{lstlisting}[style=pgsql]
CREATE TABLE t_addresses
        (
                i_id_address    serial,
                v_address       character varying(255),
                i_id_city       integer NOT NULL,
                CONSTRAINT pk_t_addresses PRIMARY KEY (i_id_address)
        )
;

CREATE TABLE t_cities
        (
                i_id_city    serial,
                v_city       character varying(255),
                CONSTRAINT pk_t_cities PRIMARY KEY (i_id_city)
        )
;

\end{lstlisting}

The main problem with this structure is the consistency between the tables. Without constraints
there is no validation for the city identifier. Invalid values will make the table's join invalid. The
same will happen if for any reason the city identifier in the table t\_cities is changed.\newline

Enforcing the relation with a foreign key will solve both of the problems.

\begin{lstlisting}[style=pgsql]
ALTER TABLE t_addresses 
  ADD CONSTRAINT fk_t_addr_to_t_city
  FOREIGN KEY (i_id_city)
  REFERENCES t_cities(i_id_city)
  ;

\end{lstlisting}

The foreign key works in two ways. When a row with an invalid i\_id\_city hits the table t\_addresses 
the key is violated and the insert aborts. Deleting or updating a row from the table t\_cities still
referenced in the table t\_addresses, violates the key as well.\newline

The enforcement is performed using the triggers. When performing a data only dump/restore, the
foreign keys will not allow the restore for some tables. The option --disable-trigger allows the restore
on the existing schema to succeed. For more information on this topic check \ref{cha:BACKUP}
and \ref{cha:RESTORE}.\newline

The many options available with the FOREIGN KEYS give us great flexibility. The referenced table can
drive different actions on the referencing data using the two event options ON DELETE\index{foreign
key, ON DELETE} and ON UPDATE\index{foreign key, ON UPDATE}. The event requires an action to perform when
fired. By default this is NO ACTION which checks the constraint only at the end of the transaction. This
is useful with the deferred keys. The other two actions are the RESTRICT which does not allow the
deferring and  the CASCADE which cascades the action to the referred rows. 

For example, let's create a foreign key restricting the delete without deferring and cascading the
updates.

\begin{lstlisting}[style=pgsql]
ALTER TABLE t_addresses 
  ADD CONSTRAINT fk_t_addr_to_t_city
  FOREIGN KEY (i_id_city)
  REFERENCES t_cities(i_id_city)
  ON UPDATE CASCADE ON DELETE RESTRICT
  ;

\end{lstlisting}

Another useful clause available only with the foreign keys and check is the NOT VALID\index{foreign
key, NOT VALID}. Creating a constraint with NOT VALID tells PostgreSQL the data is already
validated by the database developer. The initial check is then skipped and the constraint creation is
instantaneous. The constraint is then 
enforced only for the new data. The invalid constraint can be validated later with the command VALIDATE
CONSTRAINT.

\begin{lstlisting}[style=pgsql]
postgres=#ALTER TABLE t_addresses
                ADD CONSTRAINT fk_t_addr_to_t_city
                FOREIGN KEY (i_id_city)
                REFERENCES t_cities(i_id_city)
                ON UPDATE CASCADE ON DELETE RESTRICT
                NOT VALID
                ;
ALTER TABLE
postgres=# ALTER TABLE t_addresses VALIDATE CONSTRAINT fk_t_addr_to_t_city ;
ALTER TABLE

\end{lstlisting}



\section{Check constraints}
\label{sec:CHECKCNS}

A check constraint is a custom check enforcing a specific condition on the table's data.  The definition
can be a boolean expression or a used defined function returning a boolean value. Like the foreign
keys, the check accepts the NOT VALID\index{check, NOT VALID} clause.\newline

The check is satisfied if the condition returns true or NULL. This behaviour can produce unpredictable
results if not fully understood. An example will help to clarify the behaviour. Let's add
a CHECK constraint on the v\_address table in order to have no zero length addresses. The insert with
just the city succeed without key violation though.

\begin{lstlisting}[style=pgsql]
postgres=# ALTER TABLE t_addresses
                ADD CONSTRAINT chk_t_addr_city_exists
                CHECK (length(v_address)>0)
                ; 
postgres=# INSERT INTO t_cities (v_city) VALUES ('Brighton') RETURNING i_id_city;
 i_id_city 
-----------
         2

postgres=# INSERT INTO t_addresses (i_id_city) VALUES (2);
INSERT 0 1
\end{lstlisting}


This is possible because the field v\_address does not have a default value which defaults to NULL when
not listed in the insert. The check constraint is correctly violated if, for example we'll try to update
the v\_address with the empty string.

\begin{lstlisting}[style=pgsql]
postgres=# UPDATE t_addresses SET v_address ='' ;
ERROR:  new row for relation "t_addresses" violates check constraint "chk_t_addr_city_exists"
DETAIL:  Failing row contains (3, , 2)
\end{lstlisting}

Changing the default value for the v\_address field to the empty string, will make the check constraint
working as expected.

\begin{lstlisting}[style=pgsql]
postgres=# ALTER TABLE t_addresses ALTER COLUMN v_address SET DEFAULT '';
ALTER TABLE
postgres=# INSERT INTO t_addresses (i_id_city) VALUES (2);
ERROR:  new row for relation "t_addresses" violates check constraint "chk_t_addr_city_exists"
DETAIL:  Failing row contains (4, , 2).

\end{lstlisting}
Please note the existing rows are not affected by the default value change.


\section{Not null}
The NULL value is strange. When a NULL value is stored the resulting field entry is an empty object
without any type or even meaning which doesn't consumes physical space. Without specifications when a
new field this is defined accepts the NULL values.\newline

When dealing with the NULL it's important to remind that the NULL acts like the mathematical 
zero. When evaluating an expression where an element is NULL then the entire expression becomes
NULL.\newline

As seen before the fields with NULL values are usable for the unique constraints. Otherwise the primary
key does not allow the NULL values. The NOT NULL is a column constraint which does not allow the presence
of NULL values.\newline

Actually a field with the NOT NULL the unique constraint defined is exactly what the PRIMARY KEY
enforces.\newline

For example, if we want to add the NOT NULL constraint to the field v\_address in the t\_addresses 
table the command is just.

\begin{lstlisting}[style=pgsql]
postgres=# ALTER TABLE t_addresses ALTER COLUMN v_address SET NOT NULL;
ERROR:  column "v_address" contains null values

\end{lstlisting}

In this case the alter fails because the column v\_address contains NULL values from the example 
seen in \ref{sec:CHECKCNS}. The fix is quick and easy using using the coalesce function.\newline

This function returns the first parameter not null the left. 

\begin{lstlisting}[style=pgsql]
postgres=# UPDATE t_addresses
            SET v_address='EMPTY'
            WHERE v_address IS NULL;
UPDATE 1
postgres=# ALTER TABLE t_addresses ALTER COLUMN v_address SET NOT NULL;
ALTER TABLE

\end{lstlisting}

When adding new NULLable columns is instantaneous. PostgreSQL simply adds the new attribute in the
system catalogue and manages the new tuple structure considering the new field as empty space.
When the NOT NULL constraint is enforced, adding a new field requires the DEFAULT value set as well. This
is an operation to consider carefully when dealing with large data sets because the table will be
completely rewritten. This requires an exclusive lock on the affected relation. A better way to proceed
adding a NULLable field. Afterwards the new field will be set with the expected default value. Finally a
table's update will fix the NULL values without exclusive locks. When everything is fine, finally, the
NOT NULL could be enforced on the new field.
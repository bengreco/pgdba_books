\chapter{Backup}
The hardware is subject to faults. In particular if the storage is lost the entire data 
infrastructure becomes inaccessible, sometime for good. Also human errors, like not filtered 
delete or table drop can happen. Having a solid backup strategy is then the best protection, 
ensuring the data is still recoverable. In this chapter we'll take a look only to the logical 
backup with pg\_dump. 

\section{pg\_dump at glance}
\label{sec:PGDUMP}
As seen in \ref{sub:PGDUMP}, pg\_dump is the PostgreSQL's utility for saving consistent backups. 
Its usage is quite simple and if launched without options it tries to connect to the local cluster 
with the current user sending the dump to the standard output.\newline

The pg\_dump help gives useful informations about the usage.

\begin{verbatim}
postgres@tardis:~/dump$ pg_dump --help
pg_dump dumps a database as a text file or to other formats.

Usage:
  pg_dump [OPTION]... [DBNAME]

General options:
  -f, --file=FILENAME          output file or directory name
  -F, --format=c|d|t|p         output file format (custom, directory, tar,
                               plain text (default))
  -j, --jobs=NUM               use this many parallel jobs to dump
  -v, --verbose                verbose mode
  -V, --version                output version information, then exit
  -Z, --compress=0-9           compression level for compressed formats
  --lock-wait-timeout=TIMEOUT  fail after waiting TIMEOUT for a table lock
  -?, --help                   show this help, then exit

Options controlling the output content:
  -a, --data-only              dump only the data, not the schema
  -b, --blobs                  include large objects in dump
  -c, --clean                  clean (drop) database objects before recreating
  -C, --create                 include commands to create database in dump
  -E, --encoding=ENCODING      dump the data in encoding ENCODING
  -n, --schema=SCHEMA          dump the named schema(s) only
  -N, --exclude-schema=SCHEMA  do NOT dump the named schema(s)
  -o, --oids                   include OIDs in dump
  -O, --no-owner               skip restoration of object ownership in
                               plain-text format
  -s, --schema-only            dump only the schema, no data
  -S, --superuser=NAME         superuser user name to use in plain-text format
  -t, --table=TABLE            dump the named table(s) only
  -T, --exclude-table=TABLE    do NOT dump the named table(s)
  -x, --no-privileges          do not dump privileges (grant/revoke)
  --binary-upgrade             for use by upgrade utilities only
  --column-inserts             dump data as INSERT commands with column names
  --disable-dollar-quoting     disable dollar quoting, use SQL standard quoting
  --disable-triggers           disable triggers during data-only restore
  --exclude-table-data=TABLE   do NOT dump data for the named table(s)
  --inserts                    dump data as INSERT commands, rather than COPY
  --no-security-labels         do not dump security label assignments
  --no-synchronized-snapshots  do not use synchronized snapshots in parallel jobs
  --no-tablespaces             do not dump tablespace assignments
  --no-unlogged-table-data     do not dump unlogged table data
  --quote-all-identifiers      quote all identifiers, even if not key words
  --section=SECTION            dump named section (pre-data, data, or post-data)
  --serializable-deferrable    wait until the dump can run without anomalies
  --use-set-session-authorization
                               use SET SESSION AUTHORIZATION commands instead of
                               ALTER OWNER commands to set ownership

Connection options:
  -d, --dbname=DBNAME      database to dump
  -h, --host=HOSTNAME      database server host or socket directory
  -p, --port=PORT          database server port number
  -U, --username=NAME      connect as specified database user
  -w, --no-password        never prompt for password
  -W, --password           force password prompt (should happen automatically)
  --role=ROLENAME          do SET ROLE before dump

\end{verbatim}

\subsection{Connection options}
The connection options specify the way the program connects to the cluster. All the options are 
straightforward except for the password. Is possible to avoid the password prompt or to disable it 
but the password cannot be specified on the command line. In an automated dump script this can be 
worked around exporting the variable PGPASSWORD\index{PGPASSWORD} or using the password 
file.\newline

The PGPASSWORD variable is considered not secure and shouldn't be used if untrusted users are 
accessing the server. The password file is a text file named .pgpass and stored in the home 
directory of the os user which connects to the cluster.\newline
Each file's line specify a connection using the following format.

\begin{verbatim}
hostname:port:database:username:password
\end{verbatim}

If, for example, we want to connect to the database \textit{db\_test} with the username 
\textit{usr\_test} on the host \textit{tardis} with port \textit{5432} and the password is 
\textit{testpwd}\footnote{Please, don't use this simple password on production systems but generate 
a decent password with pwgen}, the password file will contain this row

\begin{verbatim}
tardis:5432:db_test:usr_test:testpwd
\end{verbatim}

For security reasons the file will not work if group or others accessible. In order to make it work 
you should issue the command chmod go-rw .pgpass . The password file is used also by other
PostgreSQL programs like the client psql.\newline

\subsection{General options}
The general options are a set of switches used to control the backup's output and format. 

The -f followed by the FILENAME outputs the backup on file.\newline

The -F specifies the backup format and requires a second option to tell pg\_dump which format to 
use. The option can be one of those, \textit{c d t p} which corresponds to 
\textit{custom directory tar plain}.\newline

If the parameter is omitted then pg\_dump uses the p format. This outputs a SQL script which 
recreates the objects when loaded into an empty database. The format is not compressed and is 
suitable for direct load using the client psql. 

The the custom together with the directory format  is most versatile format. It offers compression 
and flexibility at restore time. The file offers the parallel restore functionality and the 
selective restore of single objects.\newline

The directory format stores the schema dump, the dump's table of contents alongside with the 
compressed data dump in the directory specified with the -f switch. Each table is saved in a 
different file and is compressed by default.From the version 9.3 this format offers the parallel 
dump functionality.\newline

The tar format stores the dump in the conservative tape archive format. This format is compatible 
with directory format, does not supports compression and have the 8 GB limit on the size of 
individual tables.\newline

The -j option specifies the number of jobs to run in parallel for dumping the data. This feature 
appeared in the version 9.3 and uses the transaction's snapshot export to give a consistent data 
snapshot to the export jobs. The switch is usable only with the directory format and only 
against PostgreSQL 9.2 and later.\newline 

The option -Z specifies the compression level for the compressed formats. The default is 5 
resulting in a dumped archive from 5 to 8 times smaller than the original database. 

The option --lock-wait-timeout is the number of milliseconds for the table's lock acquisition. 
When expired the dump will fail. Is useful to avoid the program to wait forever for a table lock 
but can result in failed backups if set too much low.

\subsection{Output options}
The output options control the way the program outputs the backup. Some of those options are 
meaningful only under specified conditions, other are quite obvious.\newline

The -a option sets the data only export. Separating schema and data have some effects at restore 
time, in particular with the performance. We'll see in the detail in \ref{cha:RESTORE} how to 
build an efficient two phase restore.\newline

The -b option exports the large objects. This is the default setting except if the -n switch is 
used. In this case the -b is required to export the large objects.\newline

The options -c and -C are meaningful only for the plain output format. They respectively add the 
DROP and CREATE command before the object's DDL. For the archive formats the same option exists for 
pg\_restore.\newline

The -E specifies the character encoding for the archive. If not set the origin database encoding 
will be used.\newline 

The -n switch is used to dump the named schema only. It's possible to specify multiple -n switches 
to select many schemas or using the wildcards. However despite the efforts of pg\_dump to get all 
the dependencies resolved, something could be missing. There's no guarantee the resulting archive 
can be successfully restored.\newline

The -N switch does the opposite of the -n switch. Skips the named schema. Accepts wildcards and 
it's possible to specify multiple schemas with multiple -N switches. When both -n and -N are given, 
the behavior is to dump just the schemas that match at least one -n switch but no -N switches. If -N 
appears without -n, then schemas matching -N are excluded from what is otherwise a normal 
dump.\newline

The -o option dumps the object id as part of the table for every table. This options should be 
used only if the OIDs are part of the design. Otherwise this option shouldn't be used.\newline

The -O have effects only on plain text exports and skips the object ownership set.\newline

The -s option dumps only the database schema.\newline


The -S option is meaningful only for plain text exports. This is the super user which disables the 
triggers if the export is performed with the option --disable-triggers. Anyway, as suggested on the 
manual, it's better to run the restoring script as superuser.\newline

The -t switch is used to dump the named table only. It's possible to specify multiple tables using 
the wildcards or specifying the -t more.\newline

The -T does the opposite, skips the named table in the dump.\newline

The switch -x skips dumping the privileges settings usually dumped as grant/revoke commands.\newline

The option --binary-upgrade is used only for the in place upgrade program pg\_upgrade. Is not 
for general usage. 


The option --column-inserts result in the data exported as INSERT commands with all the column 
names specified. Because by default the data is saved as COPY statement, using this switch will 
results in a bigger dump file and a very slow restoration. It's sole advantage is any error in the 
restore will skip just the affected row and not the entire table's load.\newline

The --disable-dollar-quoting disables the newer dollar quoting for the function's body and uses the 
standard SQL quoting.\newline

The --disable-triggers emits the triggers disable 
and re enable for the data only export. Disabling the triggers ensure the foreign keys won't fail 
for the missing referenced data. This switch is meaningful only for the plain text export.\newline

The --exclude-table-data=TABLE skips the data for the named table but dumps the table's definition. 
It follow the same rules of the -t and -T for specifying multiple tables.\newline

The --insert dumps the data as INSERT command. Same as the --column-inserts the restore is very 
slow and should be used only for reloading data into non-PostgreSQL databases.\newline

The --no-security-labels doesn't include the security labels into the dump file.\newline

The --no-synchronized-snapshots  allows the parallel export against pre 9.2 databases. Because the 
snapshot export feature is missing this means the database shall not change its status meanwhile 
the export is running. Otherwise there will be a not consistent data export. If in doubt do not 
use this option.\newline

The --no-tablespaces skips the tablespace assignments.\newline

The --no-unlogged-table-data does not export data for the unlogged relations.\newline

The --quote-all-identifiers  cause all the identifiers to be enclosed in double quotes. By default 
only the identifiers with keyword's name are quoted.\newline

The --section option specifies one of the three export's sections. The pre-data, with the 
table, the view and the function definitions. The data section where the actual table data is 
dumped as COPY or INSERTS, accordingly with the command line options. The post-data section 
where the constraint, the index and the eventual GRANT REVOKE commands are finally executed. This 
switch is meaningful only for text exports. \newline

The --serializable-deferrable uses a serializable transaction for the dump, to ensure the database 
state is consistent. The dump execution waits for a point in the transaction stream without 
anomalies to avoid the risk of the dump failing or causing other transactions to abort for the 
serialization\_failure. The option is not beneficial for a dump intended only for disaster 
recovery. It's useful for the dump used for reloading the data into a read only database which 
needs to have a consistent state compatible with the origin's database.\newline

The switch --use-set-session-authorization causes the usage of SET SESSION AUTHORIZATION 
commands instead of ALTER OWNER commands to set the objects ownership. The resulting dump is more 
standards compatible but the SET SESSION AUTHORIZATION command requires the super user privileges 
whereas ALTER OWNER doesn't.




\section{Internals}
\label{sec:PGDUMPINT}

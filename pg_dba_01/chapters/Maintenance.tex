\chapter{Maintenance}
\label{cha:MAINTENANCE}\index{Maintenance}
\begin{comment}
When a new tuple's version is generated by an update it can be put everywhere, accordingly with the 
free space map so frequents updates can result in tuples moving across the data pages many and many 
times leaving a trail of dead tuples behind them. The dead tuples are physically allocated but 
invisible causing overhead and finally bloating the table. If there's any index built on 
the table, another consequence of the tuples moving across the data files is the index entries 
shall be updated. A B-tree index entry basically carries the indexed value with the pointer to the 
page containing the tuple. When the tupe change page the index entry shall be updated. This require 
an index scan to find the 
\end{comment}
\section{vacuum}
\section{reindex}
\section{analyze}



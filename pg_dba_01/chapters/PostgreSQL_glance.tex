\chapter{PostgreSQL at glance}
PostgreSQL is a first class product with high end enterprise class level features.
This first chapter is a general review on the product with a brief talk on the database's history.

\section{Long time ago in a galaxy far far away...}

Following the works of the Berkeley's Professor  Michael Stonebraker, in the
1996 Marc G. Fournier\index{Marc G. Fournier} asked for any
volunteer interested in revamping the Postgres 95 project.

\framebox{\parbox[t][1\totalheight]{1\columnwidth}{%
{\scriptsize Date: Mon, 08 Jul 1996 22:12:19-0400 (EDT) From: \char`\"{}Marc
G. Fournier\char`\"{} <scrappy@ki.net> }{\scriptsize \par}

{\scriptsize Subject: {[}PG95]: Developers interested in improving
PG95? }{\scriptsize \par}

{\scriptsize To: Postgres 95 Users <postgres95@oozoo.vnet.net> }{\scriptsize \par}

{\scriptsize Hi... A while back, there was talk of a TODO list and
development moving forward on Postgres95 ... }{\scriptsize \par}

{\scriptsize at which point in time I volunteered to put up a cvs
archive and sup server so that making updates (and getting at the
\char`\"{}newest source code\char`\"{}) was easier to do... }{\scriptsize \par}

{\scriptsize ... Just got the sup server up and running, and for those
that are familiar with sup, the following should work (ie. I can access
the sup server from my machines using this): }{\scriptsize \par}

{\scriptsize ...............}%
}}%

The answer came from Bruce Momjian\index{Bruce Momjian},Thomas
Lockhart\index{Thomas Lockhart} e Vadim Mikheev\index{Vadim Mikheev}, the very
first PostgreSQL Global Development Team.

\section{Features}
Every time a new major release is released, new powerful features join the
rich set of the product's functionalities.
There's a small excerpt of what the latest version offer in terms of flexibility
and reliability.

\subsection{Write ahead logging}
Like any RDBMS worth of this name PostgreSQL have the write ahead logging
feature.
In short, when a data block is updated the change is saved in a reliable
location, the so called write ahead log\index{wal}\index{write ahead log}. The
effective write on the datafile is performed later. Should the database crash
the WAL is scanned and the saved blocks are replayed during the
crash recovery. PostgreSQL stores the redo records in fixed size segments,
usually 16 MB. When the wal segment is full PostgreSQL switches to a newly
created or recycled wal segment in the process called log switch. \index{log
switch}

\subsection{Point in time recovery}
\index{pitr}\index{point in time recovery}\index{log shipping}When the
log switch happens is possible to archive the previous segment in a safe
location. Taking an inconsistent copy of the data directory is possible to
restore a fully functional cluster because the
archived wal segments have all the informations to replay the physical data
blocks on the inconsistent data files. The restore can be, optionally stopped
at a given point in time. For example is possible to recover a PostgreSQL
cluster to one second before the a catastrophic happening (e.g. a table
drop). 

\subsection{Standby server and high availability}
\index{standby server}\index{high availability}The inconsistent snapshot
can be configured to stay up in continuous archive recovery. PostgreSQL 8.4
supports the warm standby\index{warm standby} configuration where the standby
server does not accept connections. From the version 9.0 is possible to
enable the hot standby\index{hot standby} configuration to access the standby
server in read only mode.

\subsection{Streaming replication}
The wal archiving doesn't work in real time. The wal shipping happens only
after the log switch and in a low activity server this can leave the standby
behind the master for a while. Using the streaming replication\index{streaming
replication} a standby server can get the wal blocks over a database connection
in almost real time.


\subsection{Transactional}
PostgreSQL fully supports the transactions and is ACID compliant.
From the version 8.0 the save points were introduced.

\subsection{Procedural languages}
Amongst the rich of feature procedural language pl/pgsql, many procedural
languages such as perl or python are available for writing database functions.
The DO keyword was introduced in the 9.1 to have anonymous function's code
blocks.

\subsection{Partitioning}
The partitioning\index{partitioning}\index{constraint exclusion}, implemented in
PostgreSQL is still very basic. The partitions are tables connected
with one empty parent table using the table's inheritance. 
Defining check constraints on the partitioned criteria the database can
exclude, querying the parent table, the partitions not affected by the where
condition.
As the physical storage is distinct for each partition and there's no global
primary key enforcement nor foreign keys can be defined on the partitioned
structure.

\subsection{Cost based optimizer}
The cost based optimizer, or CBO,\index{cost based optimizer}\index{CBO} is the
one of PostgreSQL's point of strenght.
The query execution is dynamically determined and self adapting to the
underlying data structure or the estimated amount of data affected. 
PostgreSQL supports also the genetic query optimizer GEQO.

\subsection{Multi platform support}
PostgreSQL\index{platform} nowadays supports almost any unix flavour and from
the the version 8.0 is native to Windows.

\subsection{Tablespaces}
The tablespace support permits the data files fine grain distribution on the OS
filesystems. 

\subsection{MVCC}
The way PostgreSQL keeps things consistent is the MVCC which stands for Multi
Version Concurrency Control. The mechanism is neat and efficient, offering
great advantages and one single disadvantage. We'll see in detail further but
keep in mind this important sentence. \\
There's no such thing like an update in PostgreSQL.

\subsection{Triggers}
Triggers to execute automated tasks on when DML is performed on tables and also
views are supported at any level. The events triggers are also supported.

\subsection{Views}
The read only views are well consodlidated in PostgreSQL.
In the version 9.3 was added the support for the materialized and updatable.
Also the implementation is still very basic as no incremental refresh for the
mat views nor update is possible on complex views. Anyway is still possible to
replicate this behaviour using the triggers and procedures.

\subsection{Constraint enforcement}
PostgreSQL supports primary keys and unique keys to enforce local data meanwhile
the referential integrity is guaranteed with the foreign keys.
The check constraint to validate custom data sets is also supported.

\subsection{Extension system}
PostgreSQL implements the extension system. Almost all the previously known
contrib modules are now implemented in this efficient way to add feature to the
server using a simple SQL command.

\chapter{PostgreSQL at a glance}
PostgreSQL is a first class product with enterprise class features.
This chapter is nothing but a general review on the product with a short section dedicated to the 
database's history.

\section{Long time ago in a galaxy far far away...}

Following the works of the Berkeley's Professor  Michael Stonebraker, in 1996
Marc G. Fournier\index{Marc G. Fournier} asked if there was volunteers interested
in revamping the Postgres 95 project.\newline
\begin{smallverbatim}

Date: Mon, 08 Jul 1996 22:12:19-0400 (EDT) 
From: ”Marc G. Fournier” <scrappy@ki.net>
Subject: [PG95]: Developers interested in improving PG95?
To: Postgres 95 Users <postgres95@oozoo.vnet.net>
Hi... A while back, there was talk of a TODO list and development moving forward on Postgres95 ...
at which point in time I volunteered to put up a cvs archive and sup server so that making updates 
(and getting at the ”newest source code”) was easier to do...
... Just got the sup server up and running, and for those that are familiar with sup, the following 
should work (ie. I can access the sup server from my machines using this): 

..........................
\end{smallverbatim}


At this email replied  Bruce Momjian\index{Bruce Momjian},Thomas Lockhart\index{Thomas Lockhart}, 
and Vadim Mikheev\index{Vadim Mikheev}, the very first PostgreSQL Global Development Team.\newline 

Today, after almost 20 years and millions of rows of code, PostgreSQL is a robust and reliable 
relational database. The most advanced open source database. The slogan speaks truth indeed.

\section{Features}
Each time a new major release is released it adds new features to the already 
rich feature's set. What follows is a small excerpt of the latest PostgreSQL's version 
capabilities. 

\subsection{ACID compliant}
The word ACID is an acronym for Atomicity, Consistency, Isolation and Durability. An 
ACID compliant database ensures those ules are enforced at any time. \newline
\begin{itemize}

\item The atomiticy is enforced when a transaction is ``all or nothing''. For example if a 
transaction inserts a group of new rows. If just one row violates the primary key then the entire 
transaction must be rolled back leaving the table as nothing happened.

\item The consistency ensures the database is constantly in a valid state. The database steps from 
a valid state to another valid state without exceptions.

 \item The isolation is enforced when the database status can be reached like all the concurrent 
transactions were run in serial. 

\item The durability ensures the committed transactions are saved on durable storage. In the event 
of the database crash the database must recover to last valid state.

\end{itemize}

\subsection{MVCC}
PostgreSQL ensures atomiticy consistency and isolation via the MVCC. The acronym stands 
for Multi Version Concurrency Control. The mechanism is incredibly efficient, it offers
great level of concurrency keeping the transaction's snapshots isolated and consistent. There is 
one single disadvantage in the implementation. We'll see in detail in \ref{sec:MVCC} how MVCC works 
and the reason why there's no such thing like an update in PostgreSQL.

\subsection{Write ahead logging}
The durability is implemented in PostgreSQL using the write ahead log.
In short, when a data page is updated in the volatile memory the change is saved immediately on 
a durable location, the write ahead log\index{wal}\index{write ahead log}. The page is written 
on the corresponding data file later. In the event of the database crash the write ahead log is 
scanned and all the not consistent pages are replayed on the data files.
Each segment size is usually 16 MB and their presence is automatically managed by PostgreSQL. 
The write happens in sequence from the segment's top to the bottom. When this is full PostgreSQL 
switches to a new one. When this happens there is a log switch. \index{log switch}

\subsection{Point in time recovery}
\index{pitr}\index{point in time recovery}\index{log shipping}When PostgreSQL switches to a new 
WAL this could be a new segment or a recycled one. If the old WAL is archived in a safe
location it's possible to get a copy of the physical data files meanwhile the database is running. 
The hot copy, alongside with the archived WAL segments have all the informations necessary and 
sufficient to recover the database's consistent state. The recovery by default terminates when all 
the archived data files are replayed. Anyway it's possible to stop the recover at a given point in 
time. 

\subsection{Standby server and high availability}
\index{standby server}\index{high availability}The standby server is a database configured to 
stay in continuous recovery. This way a new archived WAL file is replayed as soon as it becomes 
available. This feature was first in introduced with  PostgreSQL 8.4 as warm standby\index{warm 
standby}. From the version 9.0 PostgreSQL can be configured also in hot standby\index{hot standby}  
which allows the connections for read only queries.

\subsection{Streaming replication}
The WAL archiving doesn't work in real time. The segment is shipped only after a log switch and 
in a low activity server this can leave the standby behind the master for a while. It's  possible 
to limit the problem using the archive\_timeout parameter which forces a log swith after the given 
number of seconds. However, using the streaming replication\index{streaming replication} a standby 
server can get the wal blocks over a database connection in almost real time. This feature allows 
the physical blocks to be transmitted over a conventional database connection.


\subsection{Procedural languages}
PostgreSQL have many procedural languages. Alongside with the pl/pgsql it's possible to write the 
procedure in many other popular languages like pl/perl and pl/python. From the version 9.1 is also 
supported the anonymous function's code block with the DO keyword.

\subsection{Partitioning}
Despite the partitioning\index{partitioning}\index{constraint exclusion} implementation in
PostgreSQL is still very basic it's not complex to build an efficient partitioned structure using 
the table inheritance.\newline

Unfortunately because the physical storage is distinct for each partition, is not possible to 
have a global primary key for the partitioned structure. The foreign keys can be emulated in some 
way using the triggers.

\subsection{Cost based optimiser}
The cost based optimiser, or CBO,\index{cost based optimizer}\index{CBO} is one of the PostgreSQL's 
point of strength The execution plan is dynamically determined from the data distribution and from 
the query parameters. PostgreSQL also supports the genetic query optimizer GEQO.


\subsection{Multi platform support}
PostgreSQL\index{platform} supports almost any unix flavour, and from version 8.0 runs natively on 
Windows.

\subsection{Tablespaces}
The tablespace support permits a fine grained distribution of the data files across
filesystems. In \ref{sub:TBS-LOGICAL} and \ref{sub:TBS-PHYSICAL} we'll see how to take advantage of 
this powerful feature.

\subsection{Triggers} TODO
The trigger are well supported on the tables and views. A basic implementation of the events 
triggers is also present.

\subsection{Views}
The read only views are well consodlidated in PostgreSQL.
The version 9.3 introduced the support for the materialised and updatable views.
Unfortunately the implementation is still very basic. There is no incremental refresh for the
materialised views and the complex views don't have yet the updatable feature. In this case the 
only way to have an updatable view is to use the triggers.

\subsection{Constraint enforcement}
PostgreSQL supports primary keys and unique keys to enforce table's data. The referential integrity 
is guaranteed with the foreign keys. We'll take a look to the data integrity in \ref{cha:DATAINT}

\subsection{Extension system}
PostgreSQL from the version 9.1 implements a very efficient extension system. New features can be 
installed with a simple SQL command.

\subsection{Federated}
From PostgreSQL 9.1 is possible to have foreign tables pointing to external data sources. 
PostgreSQL 9.3 introduced also the foreign table's write functionality and the PostgreSQL's 
connector.
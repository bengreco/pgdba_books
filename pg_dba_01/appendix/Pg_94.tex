\chapter{PostgreSQL 9.4}

The new PostgreSQL major version, the 9.4 was released the 18th of December 2014. Alongside the
new developer wise features this release introduces several features making the DBA life easier.

\section{ALTER SYSTEM}
This long waited feature gives the DBA the possibility to alter the configuration parameters
using a PostgreSQL client. The parameters are validated when the command is issued. Invalid values
are spotted immediately reducing greatly the risk of having a not startable instance because syntax
errors.

For example, the following command set wal\_level to hot\_standby. 
\begin{lstlisting}[style=pgsql]
 ALTER SYSTEM SET wal_level = hot_standby;

\end{lstlisting}

\section{autovacuum\_work\_mem}
This setting sets the maintenance work memory only for the autovacuum workers. As seen in
\ref{sec:VACUUM} the vacuum performance is affected by the maintenance\_work\_mem. This new
parameter allows a better flexibility in the automatic vacuum tuning.


\section{huge\_pages}
This parameter enables or disables the use of huge memory pages on Linux. Turning on this parameter can
result in a reduced CPU usage for managing large amount of memory on Linux.

\section{Replication Slots}
With the replication slots the master becomes aware of the slave's replication status. The master
with replication slots allocated does not remove the WAL segments until they have been received by all
the standbys. Also the master does not remove the rows which could cause a recovery conflict even when
the standby is disconnected. 

\section{Planning time}
Now EXPLAIN ANALYZE shows the time spent by the planner to build the execution plan, giving to the
performance tuners a better understanding of the query efficiency.

\section{pg\_prewarm}
This additional module loads the relation's data into the shared buffer after a shutdown. This allows
the cluster reaching the efficiency quickly.


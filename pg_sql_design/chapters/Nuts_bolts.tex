\chapter{Nuts and bolts}
Before starting with the book's topic I want to explain how to set up an
efficient environment and some good practice which can improve the code's readability and quality.
As somebody will notice these methods are completely opposite to the general code style trends. I'll 
try to give the motivation for each rule. Anyway, in general because the SQL universe is a strange 
place this requires strange approach. In order to write and read effectively the SQL the coder should gain 
the capability to get a mental map  between the query's sections and the underlying logic. This can be 
facilitated using a clear and well defined formatting.\newline

\section{Code Indention}

\section{The Hungarian notation}


\section{The editor}
Unlikely many commercial RDBMS PostgreSQL, ships only with the command line client psql. There is a 
good quantity of third party clients with good support for the database features and a good connectivity 
layer. An exhaustive list of those clients can be found on the PostgreSQL wiki\newline
\href{https://wiki.postgresql.org/wiki/Community\_Guide\_to\_PostgreSQL\_GUI\_Tools}{
https://wiki.postgresql.org/wiki/Community\_Guide\_to\_PostgreSQL\_GUI\_Tools}. Is difficult to say 
which editor is the best. When I started learning PostgreSQL the only tool available were PGAdmin 2 and 
phpPgAdmin. I decided for the former and I welcomed the newer version PgAdmin 3. However I tested some of 
the other clients like TOra, SQL workbench and SQL Maestro and I never found the same confidence and ease 
of usage like PgAdmin 3. Whether is the tool of your choice this should have the following features.

\subsection{Native connector}
One of the reasons I do not like SQL workbench is the JDBC connector. For me writing and testing the SQL 
code is a quick process. I write the code which is run against the database, then I update the query, 
another run and so on. The client response in this method is absolutely important. The native connector 
have virtually no lag, except the disk/network bandwidth.




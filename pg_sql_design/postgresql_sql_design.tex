\documentclass[oneside]{book}
\usepackage[utf8]{inputenc}
\usepackage{makeidx}
\usepackage{graphicx}
\setcounter{tocdepth}{2}
\usepackage{hyperref}
\hypersetup{pdfborder={0 0 0},colorlinks=false}
\usepackage{verbatim}
\newenvironment{smallverbatim}{\endgraf\small\verbatim}{\endverbatim}
\newenvironment{tinyverbatim}{\endgraf\scriptsize\verbatim}{\endverbatim}
\pagestyle{plain}
\usepackage{float}
\usepackage{pdfpages}
\usepackage{listings}
\usepackage[paperwidth=21.59cm,paperheight=27.94cm]{geometry}
\usepackage{xcolor}

\author{Federico Campoli}
\title{PostgreSQL  \\ SQL development and database design}


\makeindex

\lstdefinestyle{pgsql}{
  belowcaptionskip=1\baselineskip,
  breaklines=true,
  frame=l,
  language=SQL,
  showstringspaces=false,
  basicstyle=\footnotesize\ttfamily,
  keywordstyle=\bfseries\color{green!40!black},
  commentstyle=\itshape\color{purple!40!black},
  identifierstyle=\color{blue},
  stringstyle=\color{orange},
  morekeywords={VACUUM, FULL, ANALYZE, TABLESPACE,SET}
}


\begin{document}
\maketitle

\newpage{}



\tableofcontents{}

\chapter*{Preface}
I decided to start this book in order to share my knowledge and best practice, refined and built
over eleven years of database experience. The book is specifically written for the developers
interested in the SQL and the database design. Probably the entire approach will sound quite odd. I
don't make a mystery I have virtually no theoretical knowledge on the relational databases. The
entire writing is based on pure experience and the final goal of this book leave the readers with
the capability to find not conventional ways to solve the problems. This is not a cookbook.
I consider the cookbooks a nice thing for cooking. In my humble opinion for complex tasks on
complex systems such books are useless and even dangerous. 


\section*{Intended audience}
Developers, database designers, database architects, performance tuners

\section*{Book structure}
The book is divided in several parts. Each part is organised like a standalone book focusing
on a particular aspect of the language and the design. Many of the problems are
explained by example trying to build a logical approach on the practical solution. I decided to
write this book with a colloquial style. Think about me more like a mechanic wearing a motor
oil stained overall, rather a professor behind his desk. After all, working with the databases is
low level interaction with the system's infrastructure. \newline



\section*{Version and platform}
This book focus on PostgreSQL 9.4 installed on Debian GNU Linux 7.0. The concepts are almost the
same for the latest major versions. Change of compatibility are explicitly specified.

\chapter{Nuts and bolts}
Before to get knee deep with the book's topic I want to explain my working method and environment.
Some of those things are in open contrast with the general trending and probably will turn up
someone's nose. Unfortunately the SQL universe is a strange place and require strange behaviours
in order to ensure the perfect comprehension of the logic which lies behind a query.\newline

Let's start with the shopping list for the workshop, our set of working tools.

\section{The editor}
I can hear your voices. \textit{Why should I use a SQL editor if I have PUT\_IDE\_NAME\_HERE}.  


\section{The formatting}


\section{The schema}

\part{SQL}

\part{Procedural Languages}
\part{Database Design}
\part{Tuning}

\appendix

\listoffigures
\listoftables
\printindex{}
\end{document}

\documentclass[oneside]{book}
\usepackage[utf8]{inputenc}
\usepackage{makeidx}
\usepackage{graphicx}
\setcounter{tocdepth}{2}
\usepackage{hyperref}
\hypersetup{pdfborder={0 0 0},colorlinks=false}
\usepackage{verbatim}
\newenvironment{smallverbatim}{\endgraf\small\verbatim}{\endverbatim}
\newenvironment{tinyverbatim}{\endgraf\scriptsize\verbatim}{\endverbatim}
\pagestyle{plain}
\usepackage{float}
\usepackage{pdfpages}
\usepackage{listings}
\usepackage[paperwidth=21.59cm,paperheight=27.94cm]{geometry}
\usepackage{xcolor}

\author{Federico Campoli}
\title{PostgreSQL  \\ SQL development and database design}


\makeindex

\lstdefinestyle{pgsql}{
  belowcaptionskip=1\baselineskip,
  breaklines=true,
  frame=l,
  language=SQL,
  showstringspaces=false,
  basicstyle=\footnotesize\ttfamily,
  keywordstyle=\bfseries\color{green!40!black},
  commentstyle=\itshape\color{purple!40!black},
  identifierstyle=\color{blue},
  stringstyle=\color{orange},
  morekeywords={VACUUM, FULL, ANALYZE, TABLESPACE,SET}
}


\begin{document}
\maketitle

\newpage{}



\tableofcontents{}

\chapter*{Preface}
Here we go, again. 




\section*{Intended audience}
Developers

\section*{Book structure}
SQL and design.\newline


\section*{Version and platform}
This book covers the database version 9.3 on Debian GNU Linux 7.0.
References to older version or different platform are explicitly specified.

\part{SQL}
\part{Procedural Languages}
\part{Database Design}
\part{Tuning}

\appendix

\listoffigures
\listoftables
\printindex{}
\end{document}
